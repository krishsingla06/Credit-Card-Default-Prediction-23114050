\documentclass{article}
\usepackage{graphicx} % Required for inserting images

\title{fin}
\author{Krish Singla}
\date{June 2025}

\begin{document}

\maketitle

\section{Introduction}
\subsection*{1.1 What is Credit Card Default and Why is it Important?}

Credit card default occurs when a customer fails to repay their credit card balance, posing a financial risk to banks and lending institutions. Accurately predicting potential defaulters helps reduce losses, manage credit exposure, and enable proactive interventions. 

By analyzing historical repayment behavior, credit usage, and demographic data, we aim to build a machine learning model that identifies high-risk customers in advance—facilitating smarter lending decisions and strengthening financial stability.


\subsection*{1.2 Business Objective}

The goal of this project is to help financial institutions reduce credit risk by identifying customers likely to default on credit card payments in the upcoming month. Early detection of such defaulters allows lenders to minimize losses, optimize credit distribution, and preserve portfolio health.

To achieve this, we employ classification-based machine learning models enhanced with risk-aware strategies, including threshold tuning, recall-oriented metrics like F-score, and targeted feature engineering based on financial indicators such as credit utilization, delay streaks, and payment-to-bill ratios.

The model emphasizes capturing high-risk defaulters, even at the cost of some false positives---aligning with the business priority of minimizing missed defaults while maintaining efficiency.


\subsection*{1.3 Given Data}

The dataset comprises historical credit and repayment information for a large number of credit card customers. Each row represents a unique customer, with columns capturing demographic details and financial behavior over a six-month period.

\textbf{Key features include:}
\begin{itemize}
    \item \textbf{LIMIT\_BAL} – Assigned credit limit.
    \item \textbf{SEX, EDUCATION, MARRIAGE, AGE} – Demographic attributes.
    \item \textbf{BILL\_AMT1 to BILL\_AMT6} – Monthly bill amounts for the last six months.
    \item \textbf{PAY\_AMT1 to PAY\_AMT6} – Monthly repayment amounts.
    \item \textbf{PAY\_0 to PAY\_6} – Payment delay status per month (e.g., -1 = early, 0 = on time, 1 = one month late, etc.).
    \item \textbf{default payment next month} – Target variable (1 = default, 0 = no default).
\end{itemize}

The training set contains labeled data used to build and validate the model. The test set includes similar customer records without the target label, and the objective is to accurately predict their default status.

\subsection*{1.4 Problem Statement}

The goal of this project is to build a machine learning model that predicts whether a credit card customer will default on their payment in the upcoming month. Predictions will be based on historical behavior such as credit usage, repayment patterns, and demographic attributes.

This is a \textbf{binary classification} problem focused on identifying high-risk customers before they default. Accurate predictions enable financial institutions to proactively manage risk, minimize losses, and enhance credit decision-making.

The model should be optimized not just for accuracy, but also for \textbf{recall} and \textbf{F2-score}, to ensure that most defaulters are caught—even if it means allowing some false positives.

\subsection*{1.5 Approach}

Our solution to the credit card default prediction task was built through a structured pipeline involving data analysis, feature engineering, model development, and evaluation. The key steps were:

\begin{itemize}
    \item \textbf{Step 1: Data Exploration} – Performed exploratory analysis to understand feature distributions, detect patterns linked to default behavior, and assess class imbalance or missing values.
    
    \item \textbf{Step 2: Feature Engineering} – Created domain-informed features like credit utilization, payment-to-bill ratios, delay streaks, and repayment trends to better represent customer behavior.
    
    \item \textbf{Step 3: Handling Class Imbalance} – Addressed skewed class distribution using techniques such as SMOTE and class weighting to improve sensitivity to defaulters.
    
    \item \textbf{Step 4: Model Training \& Evaluation} – Trained multiple models (Logistic Regression, Random Forest, LightGBM, XGBoost) and evaluated them using recall, F2-score, and AUC-ROC.
    
    \item \textbf{Step 5: Threshold Optimization} – Adjusted prediction thresholds to maximize the F2-score, prioritizing recall over precision.
    
    \item \textbf{Step 6: Feature Selection} – Leveraged feature importance scores from tree-based models to eliminate low-impact variables and enhance generalization.
    
    \item \textbf{Step 7: Final Prediction} – Used the best-performing model (XGBoost with optimized threshold and selected features) to predict defaults on the test set.
\end{itemize}

\section*{2. Data Preprocessing}

Prior to model training, several preprocessing steps were applied to clean and prepare the dataset:

\begin{itemize}
    \item \textbf{Missing Values:} The \texttt{AGE} column had 126 missing entries. These were imputed using the \textbf{median age} to maintain distribution symmetry and avoid bias.

    \item \textbf{Inconsistent Categorical Entries:} Some categorical columns included invalid or inconsistent values:
    \begin{itemize}
        \item In the \texttt{MARRIAGE} column, values of \texttt{0} (an invalid category) were replaced with \texttt{3} (Others).
        \item In the \texttt{EDUCATION} column, values \texttt{0}, \texttt{5}, and \texttt{6} were also invalid. These were grouped and replaced with \texttt{4} (Others). Since these accounted for over 1\% of the data (311 entries), correction was preferred over removal.
    \end{itemize}

    \item \textbf{Invalid Values in Precomputed Columns:} Two columns in the original dataset contained \textbf{negative values} that were logically inconsistent.
\end{itemize}


\subsection*{3.1 Demographic Analysis}

Demographic features such as marriage, sex, education, and age were analyzed to explore their distribution and relationship with credit default behavior:

\begin{itemize}
    \item \textbf{Marriage:} The data includes three categories — 1 (Married), 2 (Single), and 3 (Others). A slightly higher default rate was observed among married customers across all categories.

    \item \textbf{Sex:} Female customers show a slightly higher proportion of defaults compared to male customers.

    \item \textbf{Education:} Education levels are coded as 1 (Graduate School), 2 (University), 3 (High School), and 4 (Others). Default rates are highest for customers with a High School education, followed by University and Graduate School. The “Others” group has the lowest default rate.

    \item \textbf{Age:} A box plot revealed age outliers, which were clipped using the IQR method. A KDE plot of age vs. default status shows that customers aged \textbf{25–35} have the highest default rate. The likelihood of default decreases with increasing age, indicating that younger customers are more prone to default.
\end{itemize}

\subsection*{3.2 Financial Analysis}

We conducted a detailed analysis of key financial features including credit limits, billing amounts, payments, and payment delay history to uncover their relationship with default behavior:

\begin{itemize}
    \item \textbf{Credit Limit (LIMIT\_BAL):} \\
    KDE analysis reveals that customers with lower credit limits (especially under ₹100{,}000) are more prone to default. The distribution is heavily skewed toward lower limits, with a long tail of high-credit customers. Although these upper outliers exist, we retained them due to their potential significance—representing high-value or premium users.

    \item \textbf{Monthly Bill Amounts (BILL\_AMT1 to BILL\_AMT6):} \\
    Across all six months, defaulters generally show lower bill amounts compared to non-defaulters. This may imply conservative credit usage or financial constraints among high-risk customers.

    \item \textbf{Average Bill Amount (AVG\_BILL\_AMT):} \\
    The trend continues with the average bill amount over six months—defaulters consistently spend less on average. While the data is left-skewed and contains over 2{,}000 high-value outliers, we preserved them to capture real-world financial extremes.

    \item \textbf{Monthly Payment Amounts (PAY\_AMT1 to PAY\_AMT6):} \\
    Defaulters tend to make significantly smaller payments than non-defaulters, suggesting either reduced ability or willingness to repay credit balances. This is a key differentiator in risk profiling.

    \item \textbf{Payment Delay Status (PAY\_0 to PAY\_6):} \\
    Delay history strongly correlates with default risk. Defaulters show a higher frequency of delayed payments across all six months, reinforcing that sustained delinquency is a major red flag for credit risk.

    \item \textbf{Payment-to-Bill Ratio:} \\
    This derived feature, representing how much of the billed amount was actually repaid, shows a clear pattern: defaulters typically have lower ratios, indicating partial or insufficient repayments. This behavior aligns closely with increasing financial stress.
\end{itemize}

\noindent\textbf{Insight:} Financial behaviors like consistent low repayments, high delay frequencies, and low payment-to-bill ratios are stronger indicators of default than the absolute value of bills or payments alone. These patterns help surface hidden risk in seemingly stable accounts.


\end{document}
